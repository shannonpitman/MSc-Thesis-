\chapter{Recommendations}
\label{chap:recommendations}

The recommendations chapter translates your engineering findings into actionable guidance for implementation and future development. This chapter serves as the practical bridge between technical conclusions and real-world application, providing stakeholders with clear direction for next steps based on evidence from your investigation.

Unlike conclusions, which summarise what was achieved, recommendations focus on what should be done next. This chapter demonstrates engineering judgment by prioritising actions based on technical merit, feasibility, and potential impact while acknowledging resource constraints and implementation risks.

\section{Implementation Recommendations}

Present specific, actionable recommendations for applying your findings in practice. Each recommendation should include:

\begin{itemize}
\item Clear description of the proposed action or change.
\item Technical justification based on your results.
\item Expected benefits and performance improvements.
\item Implementation requirements and resource needs.
\item Potential risks and mitigation strategies.
\item Priority level and timeline considerations.
\end{itemize}

For example: ``Implement the optimised control algorithm  to achieve the demonstrated 7.3\% efficiency improvement, requiring firmware updates and 2-week validation testing."

\section{Future Work}

Identify high-impact opportunities for extending your investigation, prioritised by potential value and feasibility. Focus on:

\begin{itemize}
\item Critical gaps that emerged from your limitations analysis.
\item Promising research directions suggested by your findings.
\item Validation studies needed for broader application.
\item Scaling considerations for different operating conditions.
\item Integration challenges with existing systems.
\end{itemize}

Provide sufficient detail for others to understand the scope and approach for each recommended investigation, while maintaining focus on engineering priorities rather than academic completeness.
