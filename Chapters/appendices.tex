\appendix

\chapter{Miscellaneous Technical Writing Standards}
\label{app:writing-standards}

Unfortunately, there are many technical writing pseudo-standards that are expected by the academic and professional world. This appendix provides examples of some of these standards which should be followed in your writing.

\section{Figures}

Figures should be used anywhere in the report where they may be valuable, except for the Abstract. Figures should be used to \textit{supplement the text}, not replace it. Figures should be used to \textit{illustrate the text}, not to explain it. Figures should be used to \textit{make the text more readable}, not to make the text more complex. Therefore, no figure should stand alone in your report. Every figure should be introduced in the text before it appears and referenced in the text afterwards (if relevant). Every figure should also be labeled with a caption. Please see the example in Figure \ref{fig:mechatronics_venn_diagram} in Chapter \ref{introduction} for how this was done there.


\section{Tables}

Tables follow the same rules as figures. Every table should be introduced in the text before it appears and referenced in the text afterwards (if relevant). Every table should also be labeled with a caption. However, for tables, the caption should be placed at the top of the table, not the bottom. Please see the example in Table \ref{tab:intro-scope} in Chapter \ref{introduction} for how this was done there.

\section{Acronyms}

Acronyms should be used consistently throughout the report. Every acronym should be defined in the text before it is used and should also be placed in the List of Acronyms. For example, in a project on UAV control, ``Inertial Measurement Unit" might be a commonly used set of words. This could be abbreviated to ``IMU" for the sake of brevity. However, if this abbreviation is used consistently throughout the report, then it should be defined in the text before it is used. For example, ``An Inertial Measurement Unit (IMU) is a sensor that measures the angular velocity and acceleration of a vehicle." After being defined, the acronym should be used in the text consistently thereafter, except for in headings, where the full name should be used.

\section{Units}

The S.I. system mandates that the unit for a quantity be placed one space after the numerical value. For an example distance measurement, ``10 m" is correct, but ``10m" is not. Notable exceptions to this rule are angles, where the degree symbol is used, and percentages, where the percent symbol is used. 

\section{Equations}

Equations that find a use more than once in a report should be labeled with a unique identifier. This identifier should be placed in parentheses after the equation, and should be used in the text to refer to the equation. For example,``The efficiency of the system is given by the equation \eqref{eq:efficiency} in Chapter \ref{chap:methodology}." 

Further, equations should be centered and properly aligned. This is important for readability and clarity. They also form part of sentence in which they are introduced, and so grammatical rules should be followed dependent on their position in the sentence. I.e. they should be followed by a period if they are the last part of a sentence, and by a comma if they are not the last part of a sentence.

\section{Paragraphs}

A single sentence is not a paragraph and should be avoided.

A paragraph is a group of sentences that are related to a \textit{single topic}. The first sentence of a paragraph should be a clear and concise statement of the topic. The subsequent sentences should develop and support this topic.