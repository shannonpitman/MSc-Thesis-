\chapter{Results}
\label{chap:results}

The results chapter presents the empirical findings of your engineering investigation in an objective, factual manner. This chapter serves as the foundation for subsequent analysis and interpretation, providing the raw evidence that supports your conclusions. The primary purpose is to document what was observed, measured, or produced during your investigation without editorial commentary or theoretical interpretation.

Effective results presentation in engineering reports requires clarity, precision, and systematic organisation. Data should be presented with appropriate statistical measures, uncertainty quantification, and clear visual representations. This chapter establishes the credibility of your work through transparent reporting of both successful outcomes and unexpected findings, including any limitations or anomalies encountered during data collection.

\section{Purpose and Principles of Results Presentation}

The results chapter fulfills several critical functions in engineering documentation:

\begin{itemize}
\item \textbf{Objective documentation}: Present findings without bias, interpretation, or speculation.
\item \textbf{Quantitative evidence}: Provide measurable data with appropriate precision and units.
\item \textbf{Systematic organisation}: Structure results logically to support the investigation objectives.
\item \textbf{Transparency}: Report all relevant findings, including unexpected results or limitations.
\item \textbf{Reproducibility}: Present sufficient detail for independent verification.
\item \textbf{Statistical rigor}: Include uncertainty measures, confidence intervals, and sample sizes.
\end{itemize}

\section{Typical Structure and Content}

An effective engineering results chapter typically includes the following elements, organised to align with your methodology and objectives:

\begin{itemize}
\item Data quality assessment and validation results.
\item Primary experimental or computational findings.
\item Performance metrics and comparative analyses.
\item Secondary results and sensitivity studies.
\item Observed anomalies, outliers, and error characterisation.
\item System validation and verification outcomes.
\end{itemize}

\section{Data Quality and Validation}

Begin by establishing the reliability and completeness of your dataset. Report data collection success rates, calibration results, and any preprocessing steps applied. This builds confidence in subsequent findings and demonstrates engineering rigor.

Document measurement uncertainties, systematic errors, and the effectiveness of quality control procedures implemented during data collection. Reference specific validation protocols from your methodology chapter.

\section{Primary Results}

Present your main findings using appropriate tables, figures, and quantitative summaries. Ensure all data includes proper units, significant figures, and uncertainty estimates. Use clear, descriptive captions that allow figures and tables to be understood independently.

\begin{table}[h]
\centering
\caption{System performance comparison showing key engineering metrics with measurement uncertainties.}
\begin{tabular}{l c c c}
\toprule Metric & Baseline & Optimised Design & Improvement \\
\midrule Efficiency (\%) & $85.0 \pm 1.2$ & $91.2 \pm 0.8$ & +7.3\% \\
Response Time (ms) & $12.3 \pm 0.5$ & $9.7 \pm 0.3$ & -21.1\% \\
Power Consumption (W) & $2.5 \pm 0.1$ & $2.1 \pm 0.1$ & -16.0\% \\
Reliability (MTBF, hrs) & 1250 & 1680 & +34.4\% \\
\bottomrule
\end{tabular}
\label{tab:performance-results}
\end{table}

\section{Anomalies and Limitations}

Document any unexpected findings, outliers, or deviations from predicted behaviour. Report equipment failures, data collection issues, or procedural variations that may have affected results. This transparency strengthens rather than weakens your work by demonstrating awareness of limitations.

The results chapter should provide a complete, unbiased record of your investigation's outcomes, establishing the empirical foundation for the analysis and interpretation that follows in the discussion chapter.
