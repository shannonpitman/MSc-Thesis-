%%%%%%%%%%%%%%%%%%%%%%%%%%%%%%%%%%%%%%%%%
% MechatronicSystems.Group Latex Template, based on:
% Maggi Memoir Thesis
% XeLaTeX Template
% Version 1.0 (04/09/25)
%
% This template has been downloaded from:
% http://www.LaTeXTemplates.com
%
% Original authors:
% Federico Maggi (fede@maggi.cc) with extensive modifications by:
% James Hepworth (james.hepworth@uct.ac.za)
%
% License:
% CC BY-NC-SA 3.0 (http://creativecommons.org/licenses/by-nc-sa/3.0/)
%
% Important notes:
% This template needs to be compiled with XeLaTeX.
%
% Most of the document content and packages are specified within structure.tex
% so if you need to make modifications to the template have a look there first!
%
% This template uses several fonts that are not available on most operating 
% systems by default. These are: EB Garamond, Envy Code R and 
% Optima Regular. You will either need to obtain these and install them on your
% system or change them to different fonts. Simply go to the Fonts block just
% below here and modify their names to other fonts. You can also comment them 
% out completely to use the default LaTeX font.
%
%%%%%%%%%%%%%%%%%%%%%%%%%%%%%%%%%%%%%%%%%

%----------------------------------------------------------------------------------------
%	PACKAGES AND OTHER DOCUMENT CONFIGURATIONS
%----------------------------------------------------------------------------------------

\documentclass[11pt,a4paper,oneside]{memoir} % Change font size here (allowable values are 9pt-12pt), change the paper size, specify one or two sided printing and specify whether to show trimming lines

\input{structure.tex} % Include the file containing the code defining the structure and style of the document

%------------------------------------------------
% Thesis Information
\newcommand{\themonth}{September 2025} % The month of writing    

\title{A (\emph{likely incomplete}) Guide to Writing Engineering Dissertations} % Thesis title

\renewcommand{\author}{James Hepworth} % Author name

\date{}

% Supervisor
\newcommand{\supervisor}{Claude 4 Sonnet}

\newcommand{\institution}{University of Cape Town\xspace} % University/institution name

\newcommand{\reseachgroup}{\href{http://mechatronicsystems.group}{MechatronicsSystems.Group}\xspace} % University/institution name

\newcommand{\department}{Department of Mechanical Engineering\xspace} % Department name

\newcommand{\degree}{BSc (Eng) in Mechanical and Mechatronic Engineering}

%------------------------------------------------
% Fonts

\defaultfontfeatures{Mapping=tex-text}
\setromanfont[Ligatures={Common}]{EB Garamond} % Normal document font
\setmonofont[Scale=0.8]{Envy Code R} % Mono spaced font (\texttt{})
\setsansfont[Scale=0.9]{Optima Regular} % Sans-serif font (\textsf{})

\renewcommand*{\acffont}[1]{{\normalsize\itshape #1}} % Font style for the acronym text (e.g. Do It Yourself)
\renewcommand*{\acfsfont}[1]{{\normalsize\upshape #1}} % Font style for the acronym in bracket (e.g. (DIY))

%------------------------------------------------
% Hyphenations

\hyphenation{a-no-ma-lous a-no-ma-ly amounts breaches} % Specify custom hyphenation points in words with dashes where you would like hyphenation to occur, or alternatively, don't put any dashes in a word to stop hyphenation altogether

%----------------------------------------------------------------------------------------
%	TITLE PAGE
%----------------------------------------------------------------------------------------

\renewcommand{\maketitlehooka}{}

\renewcommand{\maketitlehookb}{
\centering
\vfill
\includegraphics[width=0.6\linewidth]{Figures/UCT_MSG_wide.png}\\[0.5cm] % Institution logo
\par
\hrulefill}

\renewcommand{\maketitlehookc}{
% People details left, institution info right
\noindent
\begin{tabular*}{\textwidth}{@{} l @{\extracolsep{\fill}} r}
    \textbf{Author:} \author &  \reseachgroup \\
    \textbf{Supervisor:} \supervisor & \department\\
    \textbf{Date:} \themonth & \institution\\
\end{tabular*}

\vspace{1cm}

\emph{Submitted in partial fulfilment of the requirements for the degree of} \\
\degree
}

%----------------------------------------------------------------------------------------

\makeindex % Write an index file

\begin{document}

\begin{titlingpage}
\maketitle % Print the title page
\end{titlingpage}

\frontmatter % Use roman page numbering style (i, ii, iii, iv...) for the pre-content pages

%----------------------------------------------------------------------------------------
%	ABSTRACT
%----------------------------------------------------------------------------------------

\chapterstyle{default} % Apply thesis chapter style
\chapter*{Abstract}
%\addcontentsline{toc}{chapter}{Abstract} % Add to table of contents

Engineering students often need a clear, practical model for writing dissertations; this document provides an overview tailored to final‑year mechatronics projects and demonstrates effective use of a LaTeX thesis template. Many student reports lack consistent structure, rigorous scoping, and appropriate use of figures, tables, code listings, and citations, which impedes readability and assessment.

This work contributes an exemplar that explains what each report section should contain, offers typical subheadings, and embeds working examples of template elements to copy and adapt. The document is implemented in XeLaTeX using the \texttt{memoir} class with chapters for Introduction, Literature Review, Methodology, Results, Discussion, Conclusions, Recommendations, and Appendices, along with acronyms and bibliography support; the result is a shareable guide that doubles as a starting point for students to produce professional, evidence‑based engineering reports.
\vspace*{1cm}

\subsection*{\emph{How to write an abstract in 5 minutes?}}

\begin{flushright}
\textit{Answer the following questions in narrative form with no more than two sentences per question:}

\begin{itemize}
    \item \textit{What is the background to the project?}
    \item \textit{What is the problem that is being solved?}
    \item \textit{What is the contribution of the project to this problem?}
    \item \textit{What is the main conclusion of the project?}
\end{itemize}
\end{flushright}


\cleartoverso % Force a break to an even page

%----------------------------------------------------------------------------------------
%	TABLE OF CONTENTS
%----------------------------------------------------------------------------------------

%\renewcommand{\contentsname}{\chapnamefont\fontsize{14}{14}\selectfont Contents}
\tableofcontents* % Print the table of contents

\cleartoverso % Force a break to an even page

%----------------------------------------------------------------------------------------
%	LIST OF FIGURES
%----------------------------------------------------------------------------------------

%\renewcommand{\listfigurename}{\chapnamefont\fontsize{14}{14}\selectfont List of Figures}
\listoffigures % Print the list of figures

\cleartoverso % Force a break to an even page

%----------------------------------------------------------------------------------------
%	LIST OF TABLES
%----------------------------------------------------------------------------------------

%\renewcommand{\listtablename}{\chapnamefont\fontsize{14}{14}\selectfont List of Tables}
\listoftables % Print the list of tables

\cleartoverso % Force a break to an even page

%----------------------------------------------------------------------------------------
%	ACRONYMS
%----------------------------------------------------------------------------------------

\chapter{List of Acronyms}
\begin{acronym}\addtolength{\itemsep}{-\baselineskip}
  \acro{OCP}{Optimal Camera Placement}
  \acro{GDOP}{Geometric Dilution of Precision}
  \acro{FOV}{Field of View}
  \acro{AGP}{Art Gallery Problem}
\end{acronym} % Include a List of Acronyms section using acronyms.tex where they are defined

\cleartoverso % Force a break to an even page


%----------------------------------------------------------------------------------------
%	CONTENT CHAPTERS
%----------------------------------------------------------------------------------------

\mainmatter % Begin numeric (1,2,3...) page numbering

\chapterstyle{thesis} % Change the style of the Chapter header to that defined in structure.tex

\pagestyle{Ruled} % Include the chapter/section in the header along with a horizontal rule underneath

\chapter{Introduction}
\label{introduction}

This chapter orients the reader: It describes what the report is about, why it matters, what will be covered, and how the document is organised. Keep it clear, concise and specific to the engineering context. Use forward references to later chapters to help the reader. Keep the Introduction brief; move technical depth to later chapters.


\section{Subject and motivation for report}
State, in one or two sentences, the subject of the report and why it matters to the stakeholder. For example: ``This report describes the main sections and features of an engineering report template.'' Briefly mention the engineering motivation (performance, cost, safety, reliability, sustainability) and link it to measurable needs or constraints. The first sentence in this section should capture the essence of what this report is about.

\section{Background to investigation}
Provide context that sets the scene for the reader: prior events, relevant standards, existing systems, or operational constraints. Keep this factual and short; defer in-depth theory to the Literature Review (Chapter~\ref{chap:litreview}). Figures may be used if they are relevant to the report, for examples in a project on mechatronic design, you may want to introduce the field of mechatronics using a diagram such as Figure

\begin{figure}[b]
\centering
\includegraphics[width=.5\textwidth]{Figures/mechatronics_venn_diagram.png}
\caption{The overlap of traditional disciplines comprising the field of mechatronics }
\label{fig:mechatronics_venn_diagram}
\end{figure}

Here, it can be seen that mechatronics is an interdisciplinary field that draws on the principles of mechanics, electronics, computer science, and control systems to design and develop intelligent machines and systems. Typically, projects offered by the MechatronicSystems.Group are related to the disciplines within this diagram.

\section{Objectives of report}
State what the report will achieve. Use action verbs and make objectives measurable. For a typical mechatronics project, the specific objectives might be to:
\begin{itemize}
  \item Design and implement a control system for an autonomous mobile robot capable of obstacle avoidance and path following.
  \item Develop sensor fusion algorithms integrating \ac{IMU}, ultrasonic, and camera data for real-time navigation.
  \item Evaluate system performance through quantitative metrics including positioning accuracy (±5 cm), response time (<100 ms), and reliability (>95\% success rate).
  \item Validate the integrated mechatronic system under controlled laboratory conditions and document design trade-offs.
\end{itemize}

\section{Limitations and scope of investigation}
Clarify boundaries so the reader knows what is and is not covered. Indicate factors that influenced scope (e.g., time, cost, equipment availability, data access). Note populations, sites, subsystems, or operating conditions included and excluded. Explicitly state assumptions.

\begin{table}[h]
\centering
\caption{Typical scope boundaries for a mechatronics project; adapt to your specific investigation.}
\begin{tabular}{l l}
\toprule In scope & Out of scope \\
\midrule Indoor navigation on flat surfaces & Outdoor terrain and weather conditions \\
Obstacle detection and avoidance & Dynamic obstacle tracking \\
Real-time control at 50 Hz update rate & High-frequency vibration analysis \\
Laboratory testing environment & Industrial deployment scenarios \\
Single robot operation & Multi-robot coordination \\
Basic machine learning algorithms & Deep learning implementations \\
Prototype-level manufacturing & Production-ready design optimization \\
Safety systems for laboratory use & Commercial safety certifications \\
\bottomrule
\end{tabular}
\label{tab:intro-scope}
\end{table}

\section{Plan of development}
Explain how the report is organised so the reader can navigate it. This report follows a structured approach that builds systematically from foundational knowledge to practical outcomes. Chapter~\ref{chap:litreview} begins by synthesising relevant prior work and identifying the gap that this investigation addresses. Following this theoretical foundation, Chapter~\ref{chap:methodology} details the methods, data, and validation procedures employed in the study. The empirical findings are then presented in Chapter~\ref{chap:results}, which displays results through figures and tables without interpretation, allowing the data to speak for itself. Chapter~\ref{chap:discussion} subsequently interprets these results, comparing them to existing literature and discussing any limitations encountered. The investigation concludes with Chapter~\ref{chap:conclusions}, which answers the stated objectives concisely, followed by Chapter~\ref{chap:recommendations} that provides actionable recommendations and outlines future work opportunities.

 % Include the introduction chapter
\chapter{Literature Review}
\label{chap:litreview}

The literature review establishes the theoretical and practical foundation for your engineering investigation. This chapter demonstrates your understanding of the current state of knowledge in your field, critically evaluates existing solutions, and identifies the specific gap that justifies your project. Unlike a simple summary of sources, an engineering literature review synthesises prior work to build a compelling case for your approach and methodology.

A well-structured literature review serves multiple purposes: it positions your work within the broader engineering context, demonstrates awareness of existing solutions and their limitations, provides evidence for design decisions, and establishes credibility through engagement with peer-reviewed sources. This chapter should guide the reader from general background knowledge to the specific problem your project addresses.

\section{Typical Structure and Content}

An effective engineering literature review typically includes the following elements:

\begin{itemize}
\item \textbf{Scope and search strategy}: Define the boundaries of your review and explain how you identified relevant sources.
\item \textbf{Thematic synthesis of prior work}: Organise literature around key themes, technologies, or approaches rather than chronologically.
\item \textbf{Comparative analysis}: Systematically compare methods, performance metrics, and limitations across different approaches.
\item \textbf{Technical evaluation}: Assess the engineering merit of different solutions, including scalability, reliability, and practical constraints.
\item \textbf{Identified research or practice gap}: Clearly articulate what remains unsolved or inadequately addressed.
\item \textbf{Implications for your methodology}: Connect the literature directly to your design choices and experimental approach
\end{itemize}

\section{Citation Practice and Source Integration}

Use the \texttt{natbib} package commands to integrate sources naturally into your technical narrative. For example, \cite{ryalat_integration_2024} demonstrated the integration of mechatronic systems in Industry 4.0 applications, while comparative studies \cite{phad} have shown the benefits of cyber-physical system approaches. 

Organise complex comparisons using tables, such as Table \ref{tab:lit-taxonomy} below, to help readers understand the landscape of existing solutions:

\begin{table}[h]
\centering
\caption{Example taxonomy of mechatronic system approaches in modern engineering applications.}
\begin{tabular}{p{2.5cm}p{2.5cm}p{2.5cm}p{2.5cm}}
\toprule 
Approach & Application Domain & Key Technology & Performance Metric \\
\midrule 
Smart Manufacturing & Industrial & IoT + AI & Efficiency (\%) \\
Soft Robotics & Automation & Advanced Materials & Flexibility \\
Cyber-Physical Systems & Integration & Real-time Control & Response Time (ms) \\
\bottomrule
\end{tabular}
\label{tab:lit-taxonomy}
\end{table}

\section{Critical Analysis and Research Gap}

The literature review should culminate in a clear identification of the gap your project addresses. This gap might be a technical limitation in existing solutions, an unexplored application domain, a need for improved performance metrics, or insufficient validation under realistic operating conditions. 

Articulate how the limitations and strengths identified in the literature directly inform your project objectives and methodology choices. This connection ensures that your approach is grounded in evidence and addresses a genuine need.
 % Literature Review
\chapter{Methodology}
\label{chap:methodology}

The methodology chapter is the technical heart of an engineering report, providing a detailed account of \emph{how} the investigation was conducted. This chapter serves as a blueprint that enables other engineers to understand, evaluate, and potentially reproduce your work. It bridges the gap between the theoretical foundation established in the literature review and the empirical results that follow.

A well-crafted methodology demonstrates engineering rigor through systematic approaches, justified design decisions, and transparent documentation of procedures. It establishes credibility by showing that results were obtained through sound engineering practices rather than ad-hoc experimentation. The methodology also provides the framework for interpreting results and assessing their validity and reliability.

\section{Purpose and Scope of Engineering Methodology}

The methodology chapter fulfills several critical functions in engineering documentation:

\begin{itemize}
\item \textbf{Reproducibility}: Provides sufficient detail for independent verification and replication of results.
\item \textbf{Validation}: Demonstrates that methods are appropriate for addressing the stated objectives.
\item \textbf{Transparency}: Documents assumptions, limitations, and potential sources of error.
\item \textbf{Justification}: Explains why specific approaches were chosen over alternatives.
\item \textbf{Quality assurance}: Shows adherence to relevant standards and best practices.
\item \textbf{Risk management}: Identifies and addresses potential failure modes or uncertainties.
\end{itemize}

\section{Typical Structure and Content}

An effective engineering methodology typically includes the following elements, adapted to your specific investigation:

\begin{itemize}
\item System overview and requirements specification.
\item Experimental design and test planning.
\item Hardware and software architecture.
\item Data collection and preprocessing procedures.
\item Analysis methods and computational models.
\item Validation and verification protocols.
\item Quality control and uncertainty quantification.
\end{itemize}

\section{System Overview and Architecture}

Begin with a clear description of the overall system or approach. Use block diagrams, flowcharts, or system architecture figures to illustrate key components and their relationships.
Define system requirements, performance specifications, and design constraints. This establishes the criteria against which your methodology will be evaluated.

\section{Implementation Details}

Document the specific tools, platforms, and configurations used. Provide sufficient detail for reproduction while maintaining focus on engineering-relevant specifications. For example, use tables to summarise technical specifications as shown in Table \ref{tab:method-specs}, below.

\begin{table}[h]
\centering
\caption{Key hardware and software specifications for system implementation.}

\begin{tabular}{l l}
\toprule Component & Specification \\
\midrule MCU & 120\,MHz Cortex-M4F \\
Sensor & \textpm 2\,g MEMS accelerometer \\
OS & Ubuntu 22.04 LTS \\
Language & C/C++ and Python \\
\bottomrule
\end{tabular}
\label{tab:method-specs}
\end{table}

\section{Procedures and Protocols}

Present step-by-step procedures with specific parameters, tolerances, and measurement protocols. Reference relevant standards and best practices, such as calibration procedures per \cite{phad}. Include:

\begin{itemize}
\item Detailed experimental procedures.
\item Measurement protocols and instrumentation.
\item Data collection parameters (sampling rates, duration, conditions).
\item Quality control checkpoints.
\item Error handling and exception procedures.
\end{itemize}

\section{Data Processing and Analysis Framework}

Describe your approach to data processing, analysis methods, and computational models. Include mathematical formulations where appropriate, such as the efficiency calculation:

\begin{equation}
\eta = \frac{P_{\text{out}}}{P_{\text{in}}}.
\label{eq:efficiency}
\end{equation}

Document preprocessing steps, filtering approaches, and analytical techniques. Explain the rationale for chosen methods and any assumptions made.

\section{Validation and Verification}

Outline procedures for ensuring result validity and system verification. This might include:

\begin{itemize}
\item Calibration and measurement validation.
\item Cross-validation techniques.  
\item Sensitivity analysis.
\item Comparison with theoretical predictions or benchmarks.
\item Statistical significance testing.
\end{itemize}

\section{Assumptions and Limitations}

Explicitly state all significant assumptions and acknowledge methodological limitations. This transparency strengthens rather than weakens your work by demonstrating awareness of constraints and potential sources of uncertainty. The methodology should demonstrate that your approach is systematic, rigorous, and appropriate for addressing your stated objectives while acknowledging the practical constraints within which the work was conducted.
 % Methodology
\chapter{Results}
\label{chap:results}

The results chapter presents the empirical findings of your engineering investigation in an objective, factual manner. This chapter serves as the foundation for subsequent analysis and interpretation, providing the raw evidence that supports your conclusions. The primary purpose is to document what was observed, measured, or produced during your investigation without editorial commentary or theoretical interpretation.

Effective results presentation in engineering reports requires clarity, precision, and systematic organisation. Data should be presented with appropriate statistical measures, uncertainty quantification, and clear visual representations. This chapter establishes the credibility of your work through transparent reporting of both successful outcomes and unexpected findings, including any limitations or anomalies encountered during data collection.

\section{Purpose and Principles of Results Presentation}

The results chapter fulfills several critical functions in engineering documentation:

\begin{itemize}
\item \textbf{Objective documentation}: Present findings without bias, interpretation, or speculation.
\item \textbf{Quantitative evidence}: Provide measurable data with appropriate precision and units.
\item \textbf{Systematic organisation}: Structure results logically to support the investigation objectives.
\item \textbf{Transparency}: Report all relevant findings, including unexpected results or limitations.
\item \textbf{Reproducibility}: Present sufficient detail for independent verification.
\item \textbf{Statistical rigor}: Include uncertainty measures, confidence intervals, and sample sizes.
\end{itemize}

\section{Typical Structure and Content}

An effective engineering results chapter typically includes the following elements, organised to align with your methodology and objectives:

\begin{itemize}
\item Data quality assessment and validation results.
\item Primary experimental or computational findings.
\item Performance metrics and comparative analyses.
\item Secondary results and sensitivity studies.
\item Observed anomalies, outliers, and error characterisation.
\item System validation and verification outcomes.
\end{itemize}

\section{Data Quality and Validation}

Begin by establishing the reliability and completeness of your dataset. Report data collection success rates, calibration results, and any preprocessing steps applied. This builds confidence in subsequent findings and demonstrates engineering rigor.

Document measurement uncertainties, systematic errors, and the effectiveness of quality control procedures implemented during data collection. Reference specific validation protocols from your methodology chapter.

\section{Primary Results}

Present your main findings using appropriate tables, figures, and quantitative summaries. Ensure all data includes proper units, significant figures, and uncertainty estimates. Use clear, descriptive captions that allow figures and tables to be understood independently.

\begin{table}[h]
\centering
\caption{System performance comparison showing key engineering metrics with measurement uncertainties.}
\begin{tabular}{l c c c}
\toprule Metric & Baseline & Optimised Design & Improvement \\
\midrule Efficiency (\%) & $85.0 \pm 1.2$ & $91.2 \pm 0.8$ & +7.3\% \\
Response Time (ms) & $12.3 \pm 0.5$ & $9.7 \pm 0.3$ & -21.1\% \\
Power Consumption (W) & $2.5 \pm 0.1$ & $2.1 \pm 0.1$ & -16.0\% \\
Reliability (MTBF, hrs) & 1250 & 1680 & +34.4\% \\
\bottomrule
\end{tabular}
\label{tab:performance-results}
\end{table}

\section{Anomalies and Limitations}

Document any unexpected findings, outliers, or deviations from predicted behaviour. Report equipment failures, data collection issues, or procedural variations that may have affected results. This transparency strengthens rather than weakens your work by demonstrating awareness of limitations.

The results chapter should provide a complete, unbiased record of your investigation's outcomes, establishing the empirical foundation for the analysis and interpretation that follows in the discussion chapter.
 % Results/Findings
\chapter{Discussion}
\label{chap:discussion}

The discussion chapter transforms raw results into engineering insights by interpreting findings within the context of your objectives, theoretical framework, and practical constraints. This chapter serves as the analytical bridge between empirical observations and actionable conclusions, where you demonstrate your engineering judgment and critical thinking skills.

Unlike the results chapter, which presents objective findings, the discussion provides subjective interpretation, analysis of causality, and evaluation of engineering trade-offs. This is where you explain \emph{why} certain results occurred, assess their significance relative to design requirements, and explore the broader implications for engineering practice.

\section{Purpose and Scope of Engineering Discussion}

The discussion chapter fulfills several critical analytical functions:

\begin{itemize}
\item \textbf{Interpretation}: Explain what results mean in engineering terms and practical contexts.
\item \textbf{Contextualisation}: Compare findings with literature, standards, and industry benchmarks. 
\item \textbf{Causality analysis}: Propose mechanisms and explanations for observed phenomena.
\item \textbf{Trade-off evaluation}: Assess competing design objectives and performance compromises.
\item \textbf{Limitation assessment}: Critically evaluate constraints and validity boundaries.
\item \textbf{Engineering judgment}: Apply professional expertise to assess practical significance.
\end{itemize}

\section{Typical Structure and Content}

An effective engineering discussion typically addresses the following elements, tailored to your specific investigation:

\begin{itemize}
\item Performance evaluation against objectives and requirements.
\item Comparison with prior work and established benchmarks.
\item Analysis of design trade-offs and optimisation outcomes.
\item Sensitivity analysis and robustness assessment.
\item Identification of failure modes and limiting factors.
\item Practical implications for engineering applications.
\item Threats to validity and methodological limitations.
\end{itemize}

Some typical subsections of the discussion chapter are:

\section{Performance Analysis and Requirements Assessment}

Begin by evaluating your results against the original objectives and system requirements established in earlier chapters. Quantify how well your solution meets performance specifications and identify areas where targets were exceeded or not achieved.

For example, if Table~\ref{tab:performance-results} from your results chapter shows efficiency improvements of 7.3\%, discuss whether this meets your design targets, how it compares to theoretical predictions, and what factors contributed to this performance level.

\section{Comparative Analysis and Benchmarking}

Compare your findings with relevant literature, industry standards, and competing approaches identified in your literature review. This contextualises your work within the broader engineering landscape and demonstrates awareness of alternative solutions.

Discuss how your results advance the state of practice, identify novel contributions, and highlight areas where your approach offers advantages or faces limitations compared to existing methods.

\section{Engineering Trade-offs and Design Decisions}

Analyse the engineering trade-offs inherent in your design choices. Discuss how optimising one performance metric may have affected others, and evaluate whether the overall balance aligns with your design priorities.

For instance, improvements in efficiency might come at the cost of increased complexity, higher material costs, or reduced reliability. Assess these trade-offs in the context of your application requirements and user needs.

\section{Sensitivity and Robustness Analysis}

Examine how sensitive your results are to variations in key parameters, operating conditions, or assumptions. This analysis helps establish the reliability and practical applicability of your findings under real-world conditions.

Discuss the robustness of your solution to manufacturing tolerances, environmental variations, or usage patterns that differ from your test conditions.

\section{Limitations and Validity Assessment}

Critically evaluate the limitations of your methodology, assumptions, and results. Discuss potential sources of error, uncertainty, or bias that may affect the validity of your conclusions.

Address threats to internal validity (experimental design issues) and external validity (generalisability limitations). This demonstrates engineering maturity and helps readers appropriately interpret and apply your findings.

\section{Evidence-based Interpretation Guidelines}

Maintain rigorous connection between claims and evidence throughout your discussion:

\begin{itemize}
\item Reference specific results (e.g., Table~\ref{tab:performance-results}) to support interpretations.
\item Avoid introducing new data; focus on analysing existing results.
\item Propose mechanisms and explanations grounded in engineering principles.
\item Distinguish between correlation and causation in your analysis.
\item Acknowledge uncertainty and alternative explanations where appropriate.
\end{itemize}

\section{Summary}

The discussion chapter should demonstrate your ability to think critically about engineering problems, synthesise complex information, and provide insights that advance both theoretical understanding and practical application.
 % Discussion
\chapter{Conclusions}
\label{chap:conclusions}

The conclusions chapter provides a definitive summary of your engineering investigation, synthesising key findings into clear, actionable statements that directly address your original objectives. This chapter serves as the culmination of your technical work, transforming detailed results and analysis into concise engineering insights that demonstrate the value and impact of your investigation. As a culmination chapter, no new information should be introduced at this stage, rather it should be focussed on summarising the key findings and their implications.

Unlike the discussion chapter, which interprets and analyses findings, the conclusions chapter presents definitive statements about what has been achieved, learned, and demonstrated. It provides closure to the engineering narrative by explicitly answering the research questions posed in your introduction and quantifying the extent to which objectives have been met.

\section{Purpose and Structure of Engineering Conclusions}

The conclusions chapter fulfills several critical functions in engineering documentation:

\begin{itemize}
\item \textbf{Objective fulfillment}: Explicitly state how each original objective was addressed.
\item \textbf{Key findings synthesis}: Distill complex results into clear, actionable insights.
\item \textbf{Engineering value}: Quantify improvements, innovations, or contributions achieved.
\item \textbf{Practical impact}: Demonstrate real-world applicability and significance.
\item \textbf{Knowledge advancement}: Identify how your work extends engineering understanding. 
\item \textbf{Honest assessment}: Acknowledge limitations and areas for future development.
\end{itemize}

Some typical subsections of the conclusions chapter are:

\section{Summary of Key Findings}

Present your most significant findings as clear, evidence-based statements. Each conclusion should be directly traceable to results presented earlier and should quantify achievements where possible. For example:

\begin{itemize}
\item System efficiency was improved by 7.3\% compared to baseline design, exceeding the target improvement of 5\%.
\item Response time was reduced by 21.1\%, meeting the sub-10ms requirement for real-time applications.
\item Power consumption decreased by 16.0\%, contributing to extended operational lifetime.
\item Reliability increased by 34.4\% as measured by mean time between failures.
\end{itemize}

\section{Contributions to Engineering Practice}

Identify the specific contributions your work makes to engineering knowledge and practice. This might include:

\begin{itemize}
\item Novel design approaches or methodologies developed.
\item Performance improvements achieved through innovative solutions.
\item Validation of theoretical models or design principles.
\item Identification of optimal operating parameters or design configurations.
\item Development of tools, frameworks, or procedures for future use.
\item Insights into failure modes, limitations, or design trade-offs.
\end{itemize}

\section{Objective Achievement Assessment}

Systematically address each objective stated in your introduction, providing a clear assessment of achievement level:

\begin{itemize}
\item \textbf{Objective 1}: [State objective] - \emph{Fully achieved} through [brief summary of how].
\item \textbf{Objective 2}: [State objective] - \emph{Partially achieved} with [quantified results and limitations].
\item \textbf{Objective 3}: [State objective] - \emph{Exceeded expectations} by [quantified improvements].
\end{itemize}

\section{Limitations and Scope Boundaries}

Acknowledge the boundaries of your investigation and areas where objectives were not fully met. This demonstrates engineering maturity and helps readers understand the appropriate context for applying your findings:

\begin{itemize}
\item Scope limitations imposed by time, resource, or access constraints.
\item Technical limitations of methods, equipment, or approaches used.
\item Assumptions that may limit generalisability of results.
\item Areas where further investigation is needed for complete understanding.
\item Trade-offs accepted to achieve primary objectives.
\end{itemize}

\section{Engineering Impact and Significance}

Conclude by articulating the broader significance of your work for engineering practice, industry applications, or future research directions. Quantify the potential impact where possible and identify specific contexts where your findings could be applied.

The conclusions chapter should leave readers with a clear understanding of what has been accomplished, what it means for engineering practice, and how the work contributes to advancing the field.

\section{Summary}

The conclusions chapter should demonstrate your ability to synthesise complex findings, articulate practical implications, and provide clear, actionable guidance for future work.
 % Conclusions
\chapter{Recommendations}
\label{chap:recommendations}

The recommendations chapter translates your engineering findings into actionable guidance for implementation and future development. This chapter serves as the practical bridge between technical conclusions and real-world application, providing stakeholders with clear direction for next steps based on evidence from your investigation.

Unlike conclusions, which summarise what was achieved, recommendations focus on what should be done next. This chapter demonstrates engineering judgment by prioritising actions based on technical merit, feasibility, and potential impact while acknowledging resource constraints and implementation risks.

\section{Implementation Recommendations}

Present specific, actionable recommendations for applying your findings in practice. Each recommendation should include:

\begin{itemize}
\item Clear description of the proposed action or change.
\item Technical justification based on your results.
\item Expected benefits and performance improvements.
\item Implementation requirements and resource needs.
\item Potential risks and mitigation strategies.
\item Priority level and timeline considerations.
\end{itemize}

For example: ``Implement the optimised control algorithm  to achieve the demonstrated 7.3\% efficiency improvement, requiring firmware updates and 2-week validation testing."

\section{Future Work}

Identify high-impact opportunities for extending your investigation, prioritised by potential value and feasibility. Focus on:

\begin{itemize}
\item Critical gaps that emerged from your limitations analysis.
\item Promising research directions suggested by your findings.
\item Validation studies needed for broader application.
\item Scaling considerations for different operating conditions.
\item Integration challenges with existing systems.
\end{itemize}

Provide sufficient detail for others to understand the scope and approach for each recommended investigation, while maintaining focus on engineering priorities rather than academic completeness.
 % Recommendations

\chapterstyle{default} % Reset the chapter style back to the default used for non-content chapters



%----------------------------------------------------------------------------------------
%	REFERENCES
%----------------------------------------------------------------------------------------
\printbibliography[title={References}] % Uses biblatex/biber

%----------------------------------------------------------------------------------------
% APPENDICES
%----------------------------------------------------------------------------------------
\chapterstyle{thesis} % Change the style of the Chapter header to that defined in structure.tex
\appendix

\chapter{Miscellaneous Technical Writing Standards}
\label{app:writing-standards}

Unfortunately, there are many technical writing pseudo-standards that are expected by the academic and professional world. This appendix provides examples of some of these standards which should be followed in your writing.

\section{Figures}

Figures should be used anywhere in the report where they may be valuable, except for the Abstract. Figures should be used to \textit{supplement the text}, not replace it. Figures should be used to \textit{illustrate the text}, not to explain it. Figures should be used to \textit{make the text more readable}, not to make the text more complex. Therefore, no figure should stand alone in your report. Every figure should be introduced in the text before it appears and referenced in the text afterwards (if relevant). Every figure should also be labeled with a caption. Please see the example in Figure \ref{fig:mechatronics_venn_diagram} in Chapter \ref{introduction} for how this was done there.


\section{Tables}

Tables follow the same rules as figures. Every table should be introduced in the text before it appears and referenced in the text afterwards (if relevant). Every table should also be labeled with a caption. However, for tables, the caption should be placed at the top of the table, not the bottom. Please see the example in Table \ref{tab:intro-scope} in Chapter \ref{introduction} for how this was done there.

\section{Acronyms}

Acronyms should be used consistently throughout the report. Every acronym should be defined in the text before it is used and should also be placed in the List of Acronyms. For example, in a project on UAV control, ``Inertial Measurement Unit" might be a commonly used set of words. This could be abbreviated to ``IMU" for the sake of brevity. However, if this abbreviation is used consistently throughout the report, then it should be defined in the text before it is used. For example, ``An Inertial Measurement Unit (IMU) is a sensor that measures the angular velocity and acceleration of a vehicle." After being defined, the acronym should be used in the text consistently thereafter, except for in headings, where the full name should be used.

\section{Units}

The S.I. system mandates that the unit for a quantity be placed one space after the numerical value. For an example distance measurement, ``10 m" is correct, but ``10m" is not. Notable exceptions to this rule are angles, where the degree symbol is used, and percentages, where the percent symbol is used. 

\section{Equations}

Equations that find a use more than once in a report should be labeled with a unique identifier. This identifier should be placed in parentheses after the equation, and should be used in the text to refer to the equation. For example,``The efficiency of the system is given by the equation \eqref{eq:efficiency} in Chapter \ref{chap:methodology}." 

Further, equations should be centered and properly aligned. This is important for readability and clarity. They also form part of sentence in which they are introduced, and so grammatical rules should be followed dependent on their position in the sentence. I.e. they should be followed by a period if they are the last part of a sentence, and by a comma if they are not the last part of a sentence.

\section{Paragraphs}

A single sentence is not a paragraph and should be avoided.

A paragraph is a group of sentences that are related to a \textit{single topic}. The first sentence of a paragraph should be a clear and concise statement of the topic. The subsequent sentences should develop and support this topic.

%----------------------------------------------------------------------------------------
%	INDEX
%----------------------------------------------------------------------------------------

\printindex % Print the index

%----------------------------------------------------------------------------------------

\end{document}